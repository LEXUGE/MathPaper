\documentclass[12pt]{article}
\usepackage{fontspec}
\usepackage{amsmath}
\usepackage{amssymb}
\usepackage[BoldFont, SlantFont, CJKchecksingle]{xeCJK}
\setCJKmainfont[BoldFont={WenQuanYi Micro Hei/Bold}]{WenQuanYi Micro Hei}

\title{关于“数列”的通项解法的研究}
\author{宁波外国语学校 J1907 应侃洋}
\date{\today}

\begin{document}
  \maketitle
  \newpage
  \tableofcontents
  \newpage
    % 正文开始
    \section{摘要}
        本文主要介绍如何使用多元一次方程的方法求解数列规律题,并描述一些数列规律题的多解情况以及其产生原因。推导并证明数列通项公式的情况。\\
        其中的一些论证内容略微超过初中范围,可选读。
    \section{关键词}
      数列规律;多元一次方程;多解
    \section{推导与论证}
      \subsection{引言}
          数学中经常会遇到各种找规律的题目,它是最令笔者与同学头疼的,同时在笔者看来缺少化繁为简的数学之美。其中,“找规律填写数列”\footnote{本文简称“数列规律题”。注意,这并不是标准名称。}是找规律题目中重要且经典的一块内容。\\
          虽然题型在初中数学已经较为少见,但是深入研究之后还是会涉及一些有价值的代数内容。\\
      \subsection{研究设计}
          观察下列数列:
        \begin{equation}
          1, 2, 3, 5, 8, 13\cdots
        \end{equation}
          对于数列(1),可能是常见的斐波那契数列(Fibonacci number),不难想到其中一个答案:21。\\
          通项公式为:
        \begin{equation}
          m_{n}=m_{n-2}+m_{n-1}
        \end{equation}
          根据平面上的n个点可由n-1次函数准确拟合出来\footnote{要求它的系数矩阵是满秩的},假设$m_{n}$为在平面直角坐标系中坐标为$(n,m_{n})$的点。可以做出以下函数:
        \begin{equation}
          f(x)=a_{0}x^{0}+a_{1}x^{1}+a_{2}x^{2}+a_{3}x^{3}+a_{4}x^{4}+a_{5}x^{5}
        \end{equation}
          代入$n=x=1,2,3,4,5,6$,$m_{n}=1,2,3,5,8,13$,可以得到下列方程组:
        \begin{equation}
          \left\{
            \begin{aligned}
              a_{0}+a_{1}+a_{2}+a_{3}+a_{4}+a_{5}&=1\\
              a_{0}+2a_{1}+4a_{2}+8a_{3}+16a_{4}+32a_{5}&=2\\
              a_{0}+3a_{1}+9a_{2}+27a_{3}+81a_{4}+243a_{5}&=3\\
              a_{0}+4a_{1}+16a_{2}+64a_{3}+256_a{4}+1024a_{5}&=5\\
              a_{0}+5a_{1}+25a_{2}+75a_{3}+625_a{4}+3125a_{5}&=8\\
              a_{0}+6a_{1}+36a_{2}+216a_{3}+1296a_{4}+7776a_{5}&=13\\
            \end{aligned}
          \right.
        \end{equation}
          这是一个六元一次方程,且拥有6个方程,理论上可以消元解出,但是相应的也较为麻烦。这个时候,使用高斯消元(Gaussian Elimination)或者克莱斯曼法则(Cramer's rule)进行解决更为方便。
          根据上述,可以解得方程组(4)的解为:
        \begin{equation}
          \left\{
            \begin{aligned}
              a_{0}&=-\frac{2348}{2387}\\
              a_{1}&=\frac{386203}{143220}\\
              a_{2}&=-\frac{43663}{57288}\\
              a_{3}&=\frac{24}{2387}\\
              a_{4}&=\frac{2491}{57288}\\
              a_{5}&=-\frac{613}{143220}\\
            \end{aligned}
          \right.
        \end{equation}
          于是,对于数列$m$的第$n$项的通项公式是:\\
        \begin{equation}
          m_{n}=\frac{1}{143220}(-140880+386203n-\frac{218315}{2}n^{2}+1440n^{3}+\frac{12455}{2}n^{4}-613n^{5})
        \end{equation}
          将$n=7$代入通项公式(9),得到$m_{n}=\frac{39288}{2387}$\\
          这个数字非常的奇怪,但的确是符合规律的。也就是说,这道题目有两个答案:$m_{n}=\frac{39288}{2387}$或$m_{n}=21$\\
          类似的,还可以得到对于以下数列的答案:
        \begin{equation}
          1, 2, 4, 8, 16
        \end{equation}
          如果以$m_{n}=2^{n-1}$为通项公式,则可以得到答案$32$。
        若以以下函数作为通项公式:
        \begin{equation}
          f(x)=\frac{1}{24}(24-18x+23x^{2}-6x^{3}+x^{4})
        \end{equation}
          可得$m_{6}=31$。
      \subsection{一般形式与多解}
          对于任何一个已知$p$项的数列$m$的其中一项$m_{n}$,总能列出一个形如以下的函数:
        \begin{equation}
          m_{n}=f(n)=a_{0}+a_{1}n^{1}+\cdots+a_{p-1}n^{p-1}
        \end{equation}
          通过代入$m_{n}$与$n$,得到$p$个方程组成的方程组,可解出所有的系数$a_{1}\cdots{}a_{p-1}$。\\
          显然对于平面上的任意一组点,总存在一组以上或一组以上的函数来准确拟合。因此,对于任何一个数列,如果不规定函数的最高次幂,那么就拥有无限解。
      \subsection{特殊情况}
          值得注意的是,对于一种情况,使用本方法求出的结果会与标准答案一致。\\
          对于一个数列,如果其自身能够用一个任意次的函数表示出来,那么使用准确拟合求出的通项公式也会与其一致。\\
          这里给出证明\footnote{该部分内容涉及线性代数,有兴趣可以试着阅读,亦可跳过}:\\
        \subsubsection{特殊情况的证明}
            已知:数列$m$,已知$n$项。原通项公式为$f(x)_{1}=a_{0}+a_{1}x+a_{2}x^{2}+\cdots{}+a_{p-1}x^{p-1}$。
          设得的通项公式为$f(x)_{2}=b_{0}+b_{1}x+b_{2}x^{2}+\cdots{}+b_{n-1}x^{n-1}$。
            求证:$f(x)_{1}=f(x)_{2}$。\\
            不妨设$p<n$\\
            取任意一个x的值$x_{1}$,代入$f(x)_{1}$与$f(x)_{2}$并相减,得:
          \begin{equation}
            (b_{0}-a_{0})+(b_{1}-a_{1})x_{1}+(b_{2}-a_{2})x_{1}^{2}+\cdots+(b_{p-1}-a{p-1})x_{1}^{p-1}+\cdots+b_{n-1}x_{1}^{n-1}=0
          \end{equation}
            同理还可得:
          \begin{equation}
            (b_{0}-a_{0})+(b_{1}-a_{1})x_{2}+(b_{2}-a_{2})x_{2}^{2}+\cdots+(b_{p-1}-a{p-1})x_{2}^{p-1}+\cdots+b_{n-1}x_{2}^{n-1}=0
          \end{equation}
            代入$n-1$个$x$,以$x$为系数,可以得到一个方程组:
          \begin{equation}
            \left\{
              \begin{aligned}
                (b_{0}-a_{0})+(b_{1}-a_{1})x_{1}+(b_{2}-a_{2})x_{1}^{2}+\cdots+(b_{p-1}-a_{p-1})x_{1}^{p-1}+\cdots+b_{n-1}x_{1}^{n-1}&=0\\
                (b_{0}-a_{0})+(b_{1}-a_{1})x_{2}+(b_{2}-a_{2})x_{2}^{2}+\cdots+(b_{p-1}-a_{p-1})x_{2}^{p-1}+\cdots+b_{n-1}x_{2}^{n-1}&=0\\
                \cdots\\
                (b_{0}-a_{0})+(b_{1}-a_{1})x_{n-1}+(b_{2}-a_{2})x_{n-1}^{2}+\cdots+(b_{p-1}-a_{p-1})x_{n-1}^{p-1}+\cdots+b_{n-1}x_{n-1}^{n-1}&=0
              \end{aligned}
            \right.
          \end{equation}
            转换为增广矩阵的形式:
          \begin{equation}
            V=\left (
              \begin{array}{ccccccc|c}
                1 & x_{1} & x_{1}^{2} & \cdots & x_{1}^{p-1} & \cdots & x_{1}^{n-1} & 0 \\
                1 & x_{2} & x_{2}^{2} & \cdots & x_{2}^{p-1} & \cdots & x_{2}^{n-1} & 0 \\
                \vdots & \vdots & \vdots & \vdots & \vdots & \ddots & \vdots & 0 \\
                1 & x_{n-1} & x_{n-1}^{2}  & \cdots & x_{n-1}^{p-1}  & \cdots & x_{n-1}^{n-1} & 0\\
              \end{array}
            \right)
          \end{equation}
            $\because V$为范德蒙矩阵\\
            $\therefore det(V)=\prod\limits_{1\leqslant{}j\textless{}i\leqslant{}n-1}(x_{i}-x_{j}) (n\geqslant{}3)$\\
            $\because x_{i}\neq{}x_{j}$\\
            $\therefore det(V)\neq0$\\
            $\therefore$方程有唯一零解\\
            $\therefore f(x)_{1}=f(x)_{2}$\\
        \\
        根据证明,对于这种数列,存在也仅存在一个唯一的多项式通项公式。
      \section{总结}
          对于任意一个只给定前若干项的数列,其通项公式有2种可能:\\
            1. 形如$m_{n}=a_{0}+a_{1}n+a_{2}n_{2}+\cdots+a_{n-1}n_{n-1}$的线性公式。\\
            2. 特殊的通项公式:如递推公式等(非线性公式)。\\
          如果数列本身不是线性公式写得的,那理论上它存在无数种通项公式,这也意味着它有着无数的解\footnote{在不规定最高次幂的情况下可解得无数的1类通项公式}。\\
          由于非线性的特殊性,不能得到其普遍规律。但是可以肯定的是,如果数列本身不由形如$m_{n}=a_{0}+a_{1}n+a_{2}n_{2}+\cdots+a_{n-1}n_{n-1}$的多项式通项公式写得的,那它一定只存在一个1类公式\footnote{证明见3.4.1}。
    % 正文结束

\end{document}
